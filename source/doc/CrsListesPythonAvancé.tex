\documentclass[9pt]{beamer}

\usepackage[utf8]{inputenc}
\usepackage{eurosym}
\RequirePackage[francais]{babel}
%\usepackage{url}
%\usepackage{etex}
%\usepackage{enumitem}
%\usepackage{multicol}
\usepackage{xcolor}
%\usepackage{bbm}
%\usepackage{amsmath,amsthm,amssymb}
%\usepackage[official]{eurosym}
%\usepackage{pifont}
%\usepackage{exercise}
%\usepackage{graphics}
%\usepackage{array,multirow,makecell}
\usepackage{verbatim}
%\usepackage[dvipsnames]{pstricks}
\usepackage{pstricks-add,pst-plot,pst-text,pst-tree,pst-eps,pst-fill,pst-node,pst-math,pst-blur,pst-func}
%\usepackage{pgf,tikz}
%\usepackage{tipfr}
%\usepackage{thmbox}
%\usepackage{calc}
%\usepackage{ifthen}
%\usepackage{pdfpages}
%\usepackage{colortbl}
%\usepackage{sagetex}
%\usetikzlibrary{arrows,patterns}
%\input tabvar
%\usepackage{tkz-tab}
%\usepackage{listings}
%\usepackage[np]{numprint}
%\usepackage{fancybox,fancyhdr}
%\usepackage{thmtools}
%\usepackage{bclogo}
%\usepackage{lastpage}

\usepackage{tabularx}
\usepackage{array,multirow,makecell}
\usetheme{Madrid}
%\usetheme{Bergen}
\usecolortheme{beaver}
 
%Information to be included in the title page:
\title{Python}
\subtitle{Listes Python}
\author{Yannick CHISTEL}
\institute{Lycée Dumont d'Urville - CAEN}
\date{Décembre 2019}
 
%----------------------------------------------------------------------------------------------- 
% 							Commandes Tableaux
%-----------------------------------------------------------------------------------------------
\setcellgapes{1pt}
\makegapedcells
\newcolumntype{R}[1]{>{\raggedleft\arraybackslash }b{#1}}
\newcolumntype{L}[1]{>{\raggedright\arraybackslash }b{#1}}
\newcolumntype{C}[1]{>{\centering\arraybackslash }b{#1}}

\definecolor{vert}{rgb}{0,0,1}


\newcounter{num}
\setcounter{num}{0}
 
\begin{document}
 
\frame{\titlepage}

\begin{frame}
\frametitle{Parcourir un tableau}

\begin{block}{Tableaux et indices négatifs}
%On accède aux éléments d'un tableau avec leurs indices.

On peut accéder aux derniers éléments d'un tableau avec des indices négatifs :
\begin{itemize}
\item liste[0] : première valeur du tableau
\item liste[-1] : dernière valeur du tableau
\item liste[-2] : avant-dernière valeur du tableau
\end{itemize}

%\begin{tabular}{|L{2.2cm}*{5}{|C{1.2cm}}|}\hline
%Tableau & Valeur 1& Valeur 2 & & Valeur n\\\hline
%Indices positifs & 0 & 1 & & n-1\\\hline
%Indices négatifs & -(n-1) & &-2&-1\\\hline
%\end{tabular}
\end{block}

\begin{block}{Itérer sur les éléments}
%On accède aux éléments d'un tableau avec leurs indices. 
On peut accéder à un élément d'une liste avec une boucle for et la syntaxe suivante :

\textbf{for} e \textbf{in} liste:

\hspace{0.5cm}instruction avec e

\end{block}

\begin{exampleblock}{Exemple}
Soit t=[1,2,3,4,5,6]
\begin{enumerate}
\item \textbf{print(t[-1])} affiche la valeur 6
\item \textbf{print(t[-2])} affiche la valeur 5
\item \textbf{for} k \textbf{in} t:

\hspace{0.5cm}\textbf{print}(k, end=' ') affiche 1 2 3 4 5 6
\end{enumerate}

\end{exampleblock}

\end{frame}


\begin{frame}
\frametitle{Créer un tableau par \textbf{compréhension}}

\begin{block}{Création d'un tableau}
La construction d'un tableau par compréhension introduit la boucle for à l'intérieur des crochets du tableau à construire. La syntaxe est de la forme:

\hspace{0.5cm} [valeur \textbf{for} i \textbf{in range}(dimension du tableau)]
\end{block}

\begin{exampleblock}{Exemple}
\begin{enumerate}
\item Construire par compréhension une liste ordonnée de nombres:
\begin{enumerate}
\item[a)] [i \textbf{for} i in range(5)] construit le tableau [0, 1, 2, 3, 4]
\item[b)] [i**2 \textbf{for} i in range(5)] construit le tableau [0, 1, 4, 9, 16]
\item[c)] [2*i+1 \textbf{for} i in range(5)] construit le tableau [1, 3, 5, 7, 9]
\end{enumerate}
\item Construire un tableau à partir des valeurs d'un autre tableau :\\
t=[2,3,5,7,11,13]\\
p=[x**2 \textbf{for} x in t]\\
Le tableau p a pour valeur [4, 9, 25, 49, 121, 169] 
\item Construire un tableau dont les valeurs sont des chaines de caractères:\\
direction=['nord','sud','est','ouest']\\
DIRECTION=[e.upper() for e in direction]\\
Le tableau DIRECTION a pour valeur ['NORD', 'SUD', 'EST', 'OUEST']
\end{enumerate}
\end{exampleblock}

\end{frame}




\begin{frame}
\frametitle{Copier un tableau : \textbf{list}}

\begin{block}{Méthode}
On a vu que la copie d'un tableau ne se fait pas par affectation puisque les variables vont au final désigner le même tableau.

Pour déclarer une nouvelle variable en lui affectant les valeurs d'un tableau existant, on utilise la fonction \textbf{list} avec la syntaxe :

tableau2 = \textbf{list}(tableau1)

\end{block}

\begin{exampleblock}{Exemple}
t=[10,20,30,40,50] \\
u=\textbf{list}(t)\\
\textbf{print}(u) affiche [10,20,30,40,50]  \hspace{1cm}  \textit{le tableau est copié dans la variable u.}\\
\textbf{for} i \textbf{in range}(5):\\
\hspace{0.5cm}u[i]=u[i]/10  \hspace{1cm}  \textit{ici, chaque valeur de u est divisée par 10}\\
\textbf{print(t)} afffiche [10,20,30,40,50]  \hspace{1cm} \textit{ le tableau t n'a pas changé}\\
\textbf{print(u)} afffiche [1,2,3,4,5]  \hspace{1cm} \textit{ le tableau u a ses valeurs divisées par 10}
\end{exampleblock}

\end{frame}


\begin{frame}
\frametitle{Modifier les tableaux}

\begin{block}{Ajouter des valeurs}
Un tableau est de dimension fixée. Mais, en python, il est possible d'agrandir un tableau en ajoutant des valeurs. Deux méthodes sont possibles :
\begin{enumerate}
\item Par concaténation de deux tableaux existants avec l'opérateur + ;
\item En utilisant la fonction \textbf{append} qui permet d'ajouter une valeur en fin de tableau.
\end{enumerate}
\end{block}

\begin{exampleblock}{Exemple}
\begin{enumerate}
\item Par concaténation de tableaux :\\
t=[0,1,2]\\
u=[3,4,5]\\
s=t+u \hspace{1cm} \textit{le tableau u a pour valeur [0,1,2,3,4,5]}
\item Avec la fonction \textbf{append} :
t=[0,1,2]\\
t.append(3) \hspace{1cm} \textit{le tableau t a pour valeur [0,1,2,3]}\\
t.append(4) \hspace{1cm} \textit{le tableau t a pour valeur [0,1,2,3,4]}
\item Avec la fonction \textbf{append} et une boucle \textbf{for}:
t=[0,1,2]\\
u=[3,4,5]\\
\textbf{for} e in u:\\
\hspace{0.5cm}t.append(e) \hspace{1cm} \textit{le tableau t a pour valeur [0, 1, 2, 3, 4, 5]}
\end{enumerate}
\end{exampleblock}

\end{frame}

\begin{frame}
\frametitle{Des fonctions sur les tableaux}

\begin{block}{Présentation}
Il existe des fonctions que l'on peut appliquer sur un tableau, pour ajouter des éléments (append) et pour connaître le nombre d'éléments (len). En voici d'autres fonctions (liste non exhaustive):
\begin{itemize}
\item La fonction copy qui recopie les valeurs d'un tableau dans un autre ;\\
\item La fonction pop qui supprime la dernière valeur d'un tableau ;
\item La fonction insert qui insère une valeur pour un indice donné du tableau ;
\item La fonction remove qui suprrime une valeur pour un indice donné du tableau ;
\item La fonction sort trie le tableau dans l'ordre croissant.
\end{itemize}
\end{block}

\begin{exampleblock}{Exemple}
Soit un tableau : t=[0,1,2,3]
\begin{enumerate}
\item u=t.copy() \hspace{1cm} \textit{le tableau u est créé et a les mêmes valeurs que t}
\item u.pop() \hspace{1cm} \textit{la dernière valeur du tableau u est supprimée ; u vaut [0,1,2]}
\item u.insert(1,4) \hspace{1cm} \textit{la valeur 4 est insérée à l'indice 1 ; u vaut [0,4,1,2]}
\item u.remove(1) \hspace{1cm} \textit{si elle existe, la valeur indiquée est supprimée ; u vaut [0,4,2]}
\item u.sort() \hspace{1cm} \textit{le tableau u est trié ; u vaut [0,2,4]}
\end{enumerate}
\end{exampleblock}

\end{frame}


\begin{frame}
\frametitle{Tableau de tableaux}

\begin{block}{Présentation}
Un tableau peut contenir tout type de valeurs: des entiers (int), des chaines de caratères (string), des nombres réels (float) et aussi des tableaux !\medskip

Pour accéder à une valeur de ce tableau, on utilise un premier indice pour sélectionner le tableau où se trouve la valeur puis un second indice pour obtenir la valeur dans le tableau sélectionné.
\end{block}

\begin{exampleblock}{Exemple}
\begin{enumerate}
\item t=[[4,5],[6,7],[8,9]]\\
Le tableau t contient 3 tableaux de dimension 2 ; t est un tableau de dimension $3 \times 2$ ;\\
le tableau [4,5] est d'indice 0, le tableau [6,7] d'indice 1 et le tableau [8,9] d'indice 2 ;\\
les valeurs ont pour indice 0 et 1 pour chacun des trois tableaux ;\\

\textbf{print}(t[1][0]) \hspace{1cm} \textit{tableau d'indice 1, valeur d'indice 0 ; affiche la valeur 6}
\item On initialise un tableau de tableaux par compréhension :\\
t=[ [0]*3 for i in range(3) ] \hspace{1cm} le tableau t vaut [ [0,0,0], [0,0,0], [0,0,0] ]
\end{enumerate}

\end{exampleblock}

\end{frame}

\end{document}

